%%%%%%%%%%%%%%%%%%%%%%%%%%%%%%%%%%%%%%%%%
% Medium Length Professional CV
% LaTeX Template
% Version 2.0 (8/5/13)
%
% This template has been downloaded from:
% http://www.LaTeXTemplates.com
%
% Original author:
% Trey Hunner (http://www.treyhunner.com/)
%
% Important note:
% This template requires the resume.cls file to be in the same directory as the
% .tex file. The resume.cls file provides the resume style used for structuring the
% document.
%
%%%%%%%%%%%%%%%%%%%%%%%%%%%%%%%%%%%%%%%%%

%----------------------------------------------------------------------------------------
%	PACKAGES AND OTHER DOCUMENT CONFIGURATIONS 
%----------------------------------------------------------------------------------------
 
\documentclass{resume} % Use the custom resume.cls style 

\usepackage[left=0.4 in,top=0.3 in,right=0.4 in,bottom=0.3in]{geometry} % Document margins
\newcommand{\tab}[1]{\hspace{.2667\textwidth}\rlap{#1}}
\newcommand{\itab}[1]{\hspace{0em}\rlap{#1}}
\name{Li Gu} % Your name 
\address{4002, 386 Yonge Street, Toronto, ON, Canada} % Your address 
%\address{123 Pleasant Lane \\ City, State 12345} % Your secondary addess (optional) 
\address{+14168369227 \\ li.gu@mail.utoronto.ca} % Your phone number and email

\begin{document}  

%----------------------------------------------------------------------------------------
%	EDUCATION SECTION
%----------------------------------------------------------------------------------------

\begin{rSection}{Education}

{\bf University of Toronto} 
\\
{\em Department of Electrical and Computer Engineering}\hfill {September 2016 - June 2018}
\\
M.Eng Student, Focus: Machine Learning
\\\

{\bf Shanghai Jiaotong University}
\\
{\em Department of Electrical and Computer Engineering}\hfill {September 2011 - July 2015}
\\
BASc, Focus: Communication 
 


\end{rSection} 

% %----------------------------------------------------------------------------------------
% %	TECHNICAL STRENGTHS SECTION
% %----------------------------------------------------------------------------------------

% \begin{rSection}{skills and courses}

% \begin{tabular}{ @{} >{\bfseries}l @{\hspace{6ex}} l }  
% Skills & Python, C++, Matlab, Pytorch, Tensorflow, Caffe2, CUDA, AWS\\

% Courses & Machine Learning, Probabilistic Graphical Model, Random Process, Computer Vision,\\& 
% Bayesian Deep Learning, GPU Programming
 
% \end{tabular}   

% \end{rSection}

%----------------------------------------------------------------------------------------
%	PUBLICATION
%----------------------------------------------------------------------------------------


\begin{rSection}{ Publication } \itemsep -3pt        
 
 {Distilling the Posterior in Bayesian Neural Networks. Kaun-Chieh Wang, Paul Vicol, James Lucas, \textbf {Li Gu}, Richard Zemel, Roger Grosse. \textit{ICML, 2018}}

 {Graph-based Efficient WiFi Fingerprint Training Using Un-supervised Learning. Bo Zhao, Ling Pei, Changqing Xu, \textbf {Li Gu}. In\textit{Proceedings of the 28th International Technical Meeting of The Satellite Division of the Institute of Navigation (ION GNSS+ 2015)}}


% {Zhao Bo, Pei Ling, Xu Changqing, \textbf {Gu Li}, " Graph-based Efficient WiFi Fingerprint Training Using Un-supervised Learning,"\textit{Proceedings of the 28th International Technical Meeting of The Satellite Division of the Institute of Navigation (ION GNSS+ 2015)}Tampa, Florida, September 2015, pp. 2301-2310.}


 
\end{rSection}

%-------------------------------------------------------------------------------
%	PROJECTS

\begin{rSection}{Selected Projects}

\begin{rSubsection}{Video Object Segmentation} 
{June 2018 - Present}{}

\item Proposed an end-to-end framework including Mask-RCNN, FlowNet2 and custom matching and refine modules on video object segmentation 
\item Exploited several engineered modifications on Mask-RCNN such as deformable convolution, heavier box head, balanced cross entropy loss , which improves performance by 25 percent on per-frame instance segmentation on DAVIS2017 
\item Ranked 3rd on DAVIS 2017 leaderboard and submitting to ICCV 2019

 
\end{rSubsection}  

%------------------------------------------------

% \begin{rSubsection}{Investigating the optimization of LSTM on GPU}
% {February 2018 - Present}{}
 
% \item Developed custom CUDA kernel of LSTM with optimizations on single cell, single layer, multiple layer and on-chip memory for small mini-batch size 
% \item Conducted experiments of LSTM on DeepBench with custom kernel, CUDA Torch, and cuDNN6
% \item Conducted experiments of Neural Translation Machine with different CUDA kernels integrated into PyTorch
% \item Investigate the impact of data layouts and internal memory access pattern to further improve the memory efficiency
% \end{rSubsection}

%-------------------------------------------------


\begin{rSubsection}{Distilling the Posterior in Bayesian Neural Networks }
{October 2017 - February 2018}{}

\item Proposed a framework to distill MCMC samples from Stochastic Gradient Langevin Dynamics (SGLD) using a Generative Adversarial Netork (GAN), which incurs no loss in performance but reduces the storage overhead involved in maintaining MCMC samples
\item Implemented an optimitzer called SGLD on Bayesian Neural Networks with PyTorch
\item Conducted experiments on several Bayesian Neural Networks applications including classification, anomaly detection, active learning and stability of GANs
\item Accepted to ICML 2018

\end{rSubsection}

%-------------------------------------------------- 

\begin{rSubsection}{Discovering Team Strategies in Basketball from Spatiotemporal Data }
{Februray 2017 - May 2017}{}    

\item Created data explorations by visulizing NBA players' trajectoreis and match statistics, which designed rules to decide the start and end timestep of each offense in basketball match
\item Formulated players' trajectories into 2-D time series and exploited Dynamic Time Warping, EM algorithm and Hungarian algorithm to discover defense team's formation
\item Conducted experiments with SportVU dataset

\end{rSubsection} 

\end{rSection} 


%----------------------------------------------------------------------------------------
%	TECHNICAL STRENGTHS SECTION
%----------------------------------------------------------------------------------------

\begin{rSection}{skills and courses}

\begin{tabular}{ @{} >{\bfseries}l @{\hspace{6ex}} l }  
Skills & Python, PyTorch, Tensorflow, Matlab, CUDA, C++\\

Courses & Traditonal Machine Learning, Deep Learning, Random Process, Computer Vision,\\& 
Bayesian Deep Learning, GPU Programming
 
\end{tabular}   

\end{rSection}

% %--------------------------------------------------------------------------------------
% %   Research Publications 
% %--------------------------------------------------------------------------------------
% \begin{rSection}{ Research Publication } \itemsep -3pt        

% {\textbf{Akshay Vaishnav}, Path Lathiya, Mohit Sarvaiya"\textit{Design Optimization of Hydraulic Press Plate using Finite \\Element Analysis}"Vol. 6 - Issue 5, International Journal of Engineering Research and Applications (IJERA), \\ISSN: 2248-9622 } \hfill May 2016 \\    
 
% \end{rSection}


% %	INTERNSHIP/TRAININGS 
% %----------------------------------------------------------------------------------------

% \begin{rSection}{INTERNSHIP/TRAININGS} \itemsep -3pt  

% {\textbf{Automotive Industry Simulation Internship}, \\Expertshub,Sinhgad Institute of Engineering, Pune } \hfill June 2015 \\   
% {\textbf{Machining and Quality Control of Forged Connecting Rods}, \\Amul Group of Industries, Rajkot} \hfill February 2015 \\     

% \end{rSection}  
 
%  %-----------------------------------------------------------------------------
%  % POSITION OF RESPONSIBILITY
%  %-----------------------------------------------------------------------------
  
% \begin{rSection}{POSITION OF RESPONSIBILITY}

% \begin{rSubsection}{CAE and Powertrain Lead, Formula SAE}{August 2015 - Present}{GT Motorsports,a Formula Student Team of GTU}{}              
% \item Devised the design objectives and validation of designs through simulations and testings
% \item Concentrated on real time simulation of Exhaust System and the noise reduction of Exhaust system
% \item Part of core Design group in the team helping with various design decisions  
% \item Performed numerous simulations of various components of the car in the area of FEA and CFD segments with documentations   
% \end{rSubsection}  

% %------------------------------------------------

% \begin{rSubsection}{Head coordinator of Mechanical section at Robotics club} {July 2015 - May 2016}{Sanjaybhai Rajguru College of Engineering}{} 
% \item A college level Robotics club established by students with the aim of learning and professional skill development \\among students and peers            
% \item Lead in Mechanical work of Robotics club, working mostly with CAD and Hardware systems
% \item Team leader and active member working to develop various robots of different concepts and configurations     
% \end{rSubsection}

% \end{rSection}
  

% %--------------------------------------------------------------------------------------
% % Extra-Cirrucular
% %--------------------------------------------------------------------------------------

% \begin{rSection}{Extra-Cirrucular} \itemsep -2pt   

% \begin{itemize}
 
% \item STTP on \textbf{Life Long Research} under TEQIP-II, SVNIT, Surat \hfill February 2016 
% \item Participated in \textbf{Formula Student India}, An International FSAE competition,  
% \\Secured 9th rank overall \& 4th in Endurance \hfill January 2016  
% \item Seminar on \textbf{Introduction to Robotics and Arduino Programming}, SRCOE,Rajkot \hfill July 2015 
% \item \textbf{Junkyard}, BRIZINGER'15, a National Level Techfest,  GEC, Rajkot \hfill March-2015
% \item Seminar on \textbf{Rapid Prototyping}, COGNIZANCE 2K14, a National Level Technical Festival, \\CSPIT, Charotar  \hfill September-2014   
% \item \textbf{Rise of Machine}, PRAKARSH 9.0, a National Level Technical Symposium, SVIT, Vasad \hfill March-2014  
  
% \end{itemize}  


% \end{rSection} 

% %---------------------------------------------------------------------------------
% %  DECLARATION
% %--------------------------------------------------------------------------------

% \begin{rSection}{ Declaration  } \itemsep -3pt        

% \item I hereby declare that all the details furnished above are true to the best of my knowledge and belief.   
  
% \end{rSection}
\end{document}

